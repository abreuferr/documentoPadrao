\documentclass[a4paper,12pt]{report}
\usepackage[utf8]{inputenc}
\usepackage{lipsum} % Para texto exemplo
\usepackage{hyperref} % Links no sumário
\usepackage{indentfirst} % Identar o primeiro parágrafo

\begin{document}

% CAPA
\begin{titlepage}
    \centering
    {\Large Nome da Instituição\\[0.2cm]}
    {\Large Departamento\\[1cm]}
    {\Huge Título do Documento\\[2cm]}
    {\large Autor: Nome do Autor\\[0.5cm]}
    {\large Data: \today\\[2cm]}
    \vfill
\end{titlepage}

% RESUMO EXECUTIVO
\chapter*{Resumo Executivo}
\addcontentsline{toc}{chapter}{Resumo Executivo}
O resumo executivo deve fornecer uma visão geral concisa e clara do documento. Ele deve ser objetivo e cobrir os principais pontos que serão abordados.

\lipsum[1] % Texto exemplo, substitua com o conteúdo real

% SUMÁRIO
\tableofcontents
\newpage

% INTRODUÇÃO
\chapter{Introdução}
Nesta sessão, você irá introduzir o tópico abordado no documento.

\lipsum[2] % Texto exemplo

% PRIMEIRA SEÇÃO
\chapter{Título da Primeira Seção}
Aqui começa a primeira seção do documento.

\lipsum[3] % Texto exemplo

\section{Título da Primeira Subsessão}
Essa é uma subsessão dentro da primeira seção.

\lipsum[4] % Texto exemplo

\subsection{Título da Primeira Subsubsessão}
Essa é uma subsubsessão dentro da subsessão.

\lipsum[5] % Texto exemplo

% SEGUNDA SEÇÃO
\chapter{Título da Segunda Seção}
Aqui começa a segunda seção.

\lipsum[6] % Texto exemplo

\section{Título da Segunda Subsessão}
Essa é uma subsessão dentro da segunda seção.

\lipsum[7] % Texto exemplo

\subsection{Título da Segunda Subsubsessão}
Essa é uma subsubsessão dentro da subsessão.

\lipsum[8] % Texto exemplo

\end{document}