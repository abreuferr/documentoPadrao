\documentclass[a4paper,12pt]{report}

% Pacotes de idioma e codificação
\usepackage[utf8]{inputenc}
\usepackage[T1]{fontenc}  % Adicionado para melhor suporte a acentuação
\usepackage[brazil]{babel}

% Configuração de página e margens
\usepackage[a4paper, top=1.5cm, bottom=1.5cm, left=2cm, right=2cm]{geometry}

% Pacotes gráficos e de design
\usepackage{graphicx}
\usepackage{background}
\usepackage{float}
\usepackage[most]{tcolorbox}
\usepackage{xcolor}  % Adicionado para mais opções de cores

% Pacotes de funcionalidade
\usepackage{indentfirst}
\usepackage{booktabs}  % Adicionado para tabelas mais bonitas
\usepackage{enumitem}  % Adicionado para listas personalizáveis
\usepackage{lipsum}
\usepackage{microtype}  % Adicionado para melhor tipografia

% Pacote para quebrar linhas longas em verbatim
\usepackage{fvextra}  % Estende o pacote fancyvrb para melhor controle de verbatim
\DefineVerbatimEnvironment{Verbatim}{Verbatim}{breaklines=true, breakanywhere=true, fontsize=\scriptsize}

% Configuração dos hyperlinks - carregar por último para evitar conflitos
\usepackage{hyperref}
\hypersetup{
    colorlinks=true,
    linkcolor=blue,
    filecolor=magenta,
    urlcolor=cyan,
    pdftitle={Título do Documento},
    pdfauthor={Nome do Autor},
    pdfsubject={Assunto do Documento},
    pdfkeywords={LaTeX, documentação, relatório}
}

% Configuração da marca d'água centralizada
\backgroundsetup{
  scale=1.5,
  color=black,
  opacity=0.1,
  angle=0,
  position=current page.center,
  contents={\includegraphics[width=8cm]{img/logo.png}}
}

% Estilo personalizado para tcolorbox
\tcbset{
    enhanced,
    breakable,
    colback=gray!10,
    colframe=gray!40,
    arc=2mm,
    boxrule=0.5pt,
    fonttitle=\bfseries
}

% Informações do documento
\title{\Huge{\textbf{Título do Documento}}}
\author{\Large{Nome do Autor}}
\date{\Large{\today}}

\begin{document}

% Capa
\begin{titlepage}
    \maketitle
    \thispagestyle{empty} % Remove numeração na capa
\end{titlepage}

% Resumo Executivo
\chapter*{Resumo Executivo}
\addcontentsline{toc}{chapter}{Resumo Executivo}
\lipsum[1-2]
\clearpage

% Sumário
\tableofcontents
\clearpage

% Primeira Seção
\chapter{Introdução}
Este é o primeiro capítulo do documento. Aqui você pode descrever a introdução ao tema que será abordado.

% Subsessão 1
\section{Unix Command}
O comando abaixo é utilizado para instalar as dependecias no GNU/Linux Debian.

\begin{tcolorbox}[title=Comando OpenSSL]
\begin{Verbatim}
$ sudo apt install libjbig0 libtiff5 fontconfig-config libfontconfig1 libwxbase3.0-0v5 libpcsclite1 libccid pcscd opensc libengine-pkcs11-openssl gnupg-pkcs11-scd gnupg2
\end{Verbatim}
\end{tcolorbox}

% Subsessão 2
\section{Imagem}
A Figura \ref{fig:openbsd} mostra bla-bla-bla.

\begin{figure}[H]
    \centering
    \includegraphics[scale=0.10]{img/openbsd.png}
    \caption{Funcionamento da chamada de sistema do Sistema Operacional}
    \label{fig:openbsd}
\end{figure}

% Segunda Seção
\chapter{Desenvolvimento}
Neste capítulo, você pode descrever o desenvolvimento detalhado do assunto tratado no documento.

% Subsessão 1
\section{Subtópico 1 do Desenvolvimento}
Este é um exemplo de subtópico dentro do capítulo de desenvolvimento.

\clearpage

% Conclusão
\chapter{Conclusão}
Aqui você apresenta a conclusão do seu documento, resumindo as ideias principais e os resultados alcançados.

\clearpage

% Copyleft (sem numeração de capítulo)
\chapter*{}
\vspace*{\fill}
\begin{flushright}
    \begin{tabular}{l}
    \textbf{\underline{Propósito deste documento}} \\
    Conceitos de criptografia \\
    \\
    \textbf{\underline{Versão}} \\
    Versão 1.0 \\
    \\
    \textbf{\underline{Autor}} \\
    Caio Abreu Ferreira [\href{mailto:abreuferr@gmail.com}{abreuferr@gmail.com}] \\
    \end{tabular}
\end{flushright}

\end{document}