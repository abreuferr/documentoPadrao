\documentclass[a4paper,12pt]{report}
\usepackage[utf8]{inputenc}
\usepackage[brazil]{babel} % Pacote para português do Brasil
\usepackage{graphicx}
\usepackage{background} % Pacote para marca d'agua
\usepackage{indentfirst}
\usepackage{hyperref}
\usepackage[a4paper, left=2cm, right=2cm]{geometry}
\usepackage[most]{tcolorbox}
\usepackage{float}
\usepackage{lipsum} % Gerador de lero-lero
\usepackage{geometry}

% Configurando margens
\geometry{top=1.5cm,bottom=1.5cm,left=2.0cm,right=2.0cm}

% Configuração da marca d'água centralizada
\backgroundsetup{
  scale=1.5, % Ajusta o tamanho da imagem
  color=black,
  opacity=0.1,
  angle=0,
  position=current page.center, % Posiciona a imagem no centro da página
  vshift=0cm, % Ajuste vertical para o centro
  hshift=0cm, % Ajuste horizontal para o centro
  contents={\includegraphics[width=8cm]{img/logo.png}} % Logo centralizado com 8 cm de largura
}

% Capa
\title{Título do Documento}
\author{Nome do Autor}
\date{\today} % Exibe a data atual por extenso em português

\begin{document}

% Capa
\maketitle
\newpage

% Resumo Executivo
\chapter*{Resumo Executivo}
\addcontentsline{toc}{chapter}{Resumo Executivo}
\lipsum[1-2] % Texto de exemplo (remova e adicione seu texto)

\newpage

% Sumário
\tableofcontents

\newpage

% Primeira Seção
\chapter{Introdução}
Este é o primeiro capítulo do documento. Aqui você pode descrever a introdução ao tema que será abordado.

% Subsessão 1
\section{Unix Command}
O comando abaixo irá exibir a cadeia de certificados entre o senhasegura Cofre e a Microsoft.

\begin{tcolorbox}[width=\textwidth, enhanced, breakable]
	{\scriptsize
		\begin{verbatim}
		$ openssl s_client -showcerts -servername login.microsoftonline.com -connect login.microsoftonline.com:443
		\end{verbatim}
	}
\end{tcolorbox}

% Subsessão 2
\section{Imagem}
A Figura \ref{fig:openbsd} mostra bla-bla-bla.\\

\begin{figure}[H]
	\centering
	\includegraphics[scale=0.10]{img/openbsd.png}
	\caption{Funcionamento da chamada de sistema do Sistema Operacional}
	\label{fig:openbsd}
\end{figure}

% Segunda Seção
\chapter{Desenvolvimento}
Neste capítulo, você pode descrever o desenvolvimento detalhado do assunto tratado no documento.

% Subsessão 1
\section{Subtópico 1 do Desenvolvimento}
Este é um exemplo de subtópico dentro do capítulo de desenvolvimento.

\newpage

% Conclusão
\chapter{Conclusão}
Aqui você apresenta a conclusão do seu documento, resumindo as ideias principais e os resultados alcançados.

\newpage

% Copyleft (sem numeração de capítulo)
\chapter*{} % Capítulo vazio para remover cabeçalho
\vspace*{\fill}
\begin{flushright}
	\underline{\textit{Propósito deste documento}}\\
	Conceitos de criptografia\bigskip

	\underline{\textit{Versão}}\\
	Versão 1.0\bigskip

	\underline{\textit{Autor}}\\
	Caio Abreu Ferreira [\href{mailto:abreuferr@gmail.com}{abreuferr@gmail.com}] \bigskip\\
\end{flushright}

\end{document}