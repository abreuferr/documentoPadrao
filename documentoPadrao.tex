\documentclass[a4paper,12pt]{report}
\usepackage[utf8]{inputenc}
\usepackage{graphicx}
\usepackage{fancyhdr}
\usepackage{background}
\usepackage{indentfirst}
\usepackage{hyperref}

% Configuração da marca d'água
\backgroundsetup{
  scale=1.5, % Ajusta o tamanho da imagem
  color=black,
  opacity=0.1,
  angle=0,
  position=current page.center, % Posiciona a imagem no centro da página
  vshift=0cm, % Ajuste vertical para o centro
  hshift=0cm, % Ajuste horizontal para o centro
  contents={\includegraphics[width=8cm]{img/logo.png}} % Logo centralizado com 8 cm de largura
}

% Configurações de página
\pagestyle{fancy}
\fancyhf{}
\fancyhead[L]{\leftmark}
\fancyfoot[C]{\thepage}

% Capa
\title{Título do Documento}
\author{Nome do Autor}
\date{\today}

\begin{document}

% Capa
\maketitle
\newpage

% Resumo Executivo
\begin{abstract}
Este é o resumo executivo do documento. Aqui você descreve brevemente os principais pontos abordados, destacando os objetivos, metodologia e conclusões.
\end{abstract}
\newpage

% Sumário
\tableofcontents
\newpage

% Primeira Seção
\chapter{Introdução}
Este é o primeiro capítulo do documento. Aqui você pode descrever a introdução ao tema que será abordado.

% Subsessão 1
\section{Subtópico 1}
Aqui você pode expandir sobre o tópico com mais detalhes.

% Subsessão 2
\section{Subtópico 2}
Nesta seção, você pode discutir outra parte importante do tópico.

% Segunda Seção
\chapter{Desenvolvimento}
Neste capítulo, você pode descrever o desenvolvimento detalhado do assunto tratado no documento.

% Subsessão 1
\section{Subtópico 1 do Desenvolvimento}
Este é um exemplo de subtópico dentro do capítulo de desenvolvimento.

\newpage
% Conclusão
\chapter{Conclusão}
Aqui você apresenta a conclusão do seu documento, resumindo as ideias principais e os resultados alcançados.

\end{document}
